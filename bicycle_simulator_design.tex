\documentclass[11pt,a4paper,reqno]{amsart}
\usepackage{geometry}               % See geometry.pdf to learn the layout options. There are lots.
\usepackage[parfill]{parskip}       % Begin paragraphs with an empty line rather than an indent
\usepackage{bm}                     % bold math
\usepackage[colorlinks=false]{hyperref} % make references into hyperlinks

\geometry{left = 3cm, right = 3cm, top = 3cm, bottom = 3cm} % reduce margins
\frenchspacing                      % single space between sentences

\title{Design and validation of a static bicycle simulator}
%\author{Oliver Lee}


% define common math terms
\newcommand{\mass}{\bm{M}}
\newcommand{\damping}{v \bm{C}_1}
\newcommand{\stiffness}{g \bm{K}_0 + v^2 \bm{K}_2}

% state terms
\newcommand{\dstate}{\dot{\bm{x}}}
\newcommand{\state}{\bm{x}}
\newcommand{\sysInput}{\bm{u}}
\newcommand{\sysOutput}{y}
\newcommand{\stateMat}{\bm{A}}
\newcommand{\inputMat}{\bm{B}}
\newcommand{\outputMat}{\bm{C}}

% estimate terms
\newcommand{\dstateEst}{\dot{\hat{\bm{x}}}}
\newcommand{\stateEst}{\hat{\bm{x}}}
\newcommand{\meas}{z}
\newcommand{\processCov}{\bm{Q}}
\newcommand{\processNoise}{\bm{w}}
\newcommand{\measCov}{R}
\newcommand{\measNoise}{v}
\newcommand{\kalmanGain}{\bm{K}}
\newcommand{\estimateCov}{\bm{P}}

\begin{document}
\maketitle

\section{Introduction}
%% overview on cycling
%% address safety in cycling, understanding rider behavior
Cycling is a common form of transportation in a number of countries.
While it has long been prevalent as a low-cost form of transportation in many developing countries,
there has been a resurgence in ridership for developed countries\cite{FIXME} due to a variety of factors including,
not subject to the congestion of car traffic in large cities.
The last few years have shown a large number of single vehicle bicycle accidents\cite{FIXME}.
As in single-vehicle accidents, the cause of these accidents can be limited to bicycle handing qualities and
environmental factors.
A better understanding of the bicycle rider system is beneficial in understanding these accidents and is necessary to
reduce these in number and to improve the safety of cyclists.

%% history on bicycle and rider research
% FIXME: references used in 'Data collected for ..'
The bicycle has been used for transport since the 19th century\cite{wilson2004}, however the body of work studying the
vehicle is smaller relative to motorized vehicles such as motorcycles, cars, and planes.
Whipple developed a model for the bicycle in 1899\cite{whipple1899} but it was not until 2007 that the model was
experimentally validated\cite{kooijman2008}.
Models, as well as real world experience, show the bicycle rider system to be unstable for a range of speeds;
active control from the rider is required rider to stabilize the bicycle.
Recent work in the field of bicycle dynamics has focused on identification of rider control
models by means of experimental data collection\cite{delange2011,hess2012,hladun2015}.
Data collected for lateral line tracking\cite{delange2011,hess2012}, lane change\cite{hladun2015}, and stabilization
with roll torque disturbances\cite{delange2011,hess2012} were used for system identification and parameter estimation of
the gains of the human controller.

%% why is a simulator important (motivation for building a simulator)?
% goal of system identification
In car driver behavior research, studies have commonly used driving simulators\cite{steen2011}.
System identification has been used to evaluate performance-based driver steering models designed to match real driver
behavior\cite{pilutti1999,steen2011}.
To a smaller extent, this has also been done with motorcycle simulators have been used to study rider
behavior\cite{kovacsova2015}.
There are always limitations with simulators;
certainly with a static simulator, one is incapable of simulating the vestibular inputs.
Studies from Crundall using static simulators suggest vestibular sensory input may not be necessary for system
identification of motorcycle riders\cite{crundall2012}.
However, a simulator with high enough fidelity provides a number of benefits compared to using instrumented bicycles.
Namely, reproducibility of experiments, ease of data collection, and improved safety in test environments.
As the rider bicycle system is unstable, simulators allow for tests of scenarios that have high risk of injury if
instrumented bicycles were to be used.

There is little work focusing on human rider control using a bicycle simulator.
De Lange created a simulator in Matlab with the rider controlling the virtual bicycle by means of a
gamepad\cite{delange2011}.
His simulator only provides visual and not vestibular or proprioceptive feedback and finds its use limited for
analyzing rider control due to difficulty in controlling the bicycle.
As a precursor to this work, Schwab and Recuero prototyped a simulator with actuated handlebars\cite{schwab2013},
presenting the rider with visual and proprioceptive feedback.
In a preliminary study with two subjects, they find haptic handlebar feedback torque necessary for the rider to
stabilize the bicycle;
subjects failed to stabilize the bicycle for more than a few seconds without feedback torque.

This work describes the design of a bicycle simulator that will be used for research in system identification and human
rider control.
We begin with a description of the model used and, at a high-level, the software and hardware design of the simulator.
Then a test procedure to evaluate simulator fidelity is presented.
Simulator data and data from naturalistic studies are compared and discussed.
Finally, we end with some conclusions and suggestions for future work.

\section{Simulation of bicycle equations of motion}
\subsection{Steer Dynamics} \label{sec:steer_dynamics}
\subsubsection{Whipple Linearized Model}
The simulator uses the Whipple model\cite{whipple1899} along with a benchmark set of parameters generated by
Meijaard\cite{meijaard2007}.
The model describes the equations of motion in terms of rear assembly roll angle $\phi$ and relative front and rear
assembly steer angle $\delta$, linearized about the zero roll and zero steer configuration:
\begin{equation}
    \mass \ddot{\bm{q}} + \damping \dot{\bm{q}} + (\stiffness) \bm{q} = \bm{f} \label{eq:eom}
\end{equation}
where
$ \bm{q} = \begin{bmatrix} \phi & \delta \end{bmatrix}^T $,
$ \dot{\bm{q}} = \begin{bmatrix} \dot{\phi} & \dot{\delta} \end{bmatrix}^T $,
$ \ddot{\bm{q}} = \begin{bmatrix} \ddot{\phi} & \ddot{\delta} \end{bmatrix}^T $,
$ \bm{f} = \begin{bmatrix} T_\phi & T_\delta \end{bmatrix}^T $ with $ T_\phi $ and $ T_\delta $ are the input roll
torque and steer torque of the system.
$ \mass $ is the mass, $ \damping $ the damping, and $ \stiffness $ the stiffness of the system,
parameterized by forward velocity $ v $ and $ g $ is gravitational acceleration.

This model assumes an idealized contact between the bicycle tires and ground; no-slip knife-edge point contact.
The rider is rigidly fixed to the bicycle rear assembly; experiments have shown very little upper body lean relative to
the bicycle rear assembly\cite{kooijman2009}.
The rider can apply a steer torque $ T_\delta $ but is unable to apply a roll torque $ T_\phi $ as there is no means to
do so. At this time, there are no environmental factors that contribute a roll torque (such as crosswind) and as such,
we always take $ T_\phi = 0 $ and can rewrite the input as $ \bm{f} = \begin{bmatrix} 0  & T_\delta \end{bmatrix}^T $.
While the rider arms are necessary to provide the input steer torque $ T_\delta $, the movement of the arms is
neglected in this model.

% TODO: While Jason does mention these things in his PhD dissertation, perhaps it is better to cite another paper?
Torque sensors are commonly used to provide input to the simulated equations of motion\cite{FIXME}; the torques
necessary to steer a bicycle are small with magnitudes commonly between 0 and 3 Nm\cite{moore2012}.
While sensors that can accurately measure torques in this range do exist, any measurements may suffer from large
cross-torques induced by the rider in directions orthogonal to the steer axis.
In these cases, others have designed ways to obtain an accurate measurement of steer torque\cite{moore2012}.
However, given that we have access to a high resolution encoder to measure steer angle, we numerically compute the steer
angular acceleration $ \ddot{\delta} $ in order to estimate the rider applied steer torque $ T_\delta $.

The dynamics of the physical handlebars are described by the following equation of motion:
\begin{equation}
    I_\delta \ddot{\delta} = T_\delta - T_f
\end{equation}
where $ I_\delta $ is the moment of inertia of the handlebar about the steer axis and $ T_f $ the feedback torque
applied to the handlebars.

\subsubsection{State Estimation}
The roll angle $ \phi $ is simulated and presented to the rider via the visual environment.
Obtaining the roll angle $ \phi $ requires the plant to be observable which can be ascertained by evaluating the rank of
the observability matrix:
\begin{equation}
    \bm{\mathcal{O}} = \begin{bmatrix} \outputMat \\ \outputMat\stateMat \\
                                       \outputMat\stateMat^2 \\ \outputMat\stateMat^3 \end{bmatrix}
\end{equation}
where $ \bm{A} $ and $ \bm{C} $ are the state and output matrices of the state space form of \autoref{eq:eom}.
We measure only steer angle $ \delta $ and have the resultant state space equations:
\begin{equation}
\begin{aligned}
    \dstate &= \stateMat \state + \inputMat \sysInput \\
    \sysOutput &= \outputMat \state \label{eq:ss}
\end{aligned}
\end{equation}
where $ \dot{\bm{x}} = \begin{bmatrix} \dot{\phi} & \dot{\delta} & \ddot{\phi} & \ddot{\delta} \end{bmatrix}^T $,
$ \bm{x} = \begin{bmatrix} \phi & \delta & \dot{\phi} & \dot{\delta} \end{bmatrix}^T $,
and $ \bm{u} = \bm{f} $ from \autoref{eq:eom}.
The state space matrices are defined as:
\begin{equation}
\begin{aligned}
    \stateMat &= \begin{bmatrix} \bm{0} & \bm{I} \\
                -\mass^{-1} (\stiffness) & -\mass^{-1} (\damping) \end{bmatrix} \\
    \inputMat &= \begin{bmatrix} \bm{0} \\ \mass^{-1} \end{bmatrix} \\
    \outputMat &= \begin{bmatrix} 0 & 1 & 0 & 0 \end{bmatrix}
\end{aligned}
\end{equation}
Note that only the system is still parameterized by forward speed $ v $  as can be seen by the expression for
state matrix $ \stateMat $.

We find the observability matrix $ \bm{\mathcal{O}} $ to be full rank for all forward speeds except
$ v_1 = FIXME $ and $ v_2 = FIXME $.
The system is not observable at these velocities, however we limit the precision of the forward speed $ v $ in the
simulation to maintain an observable system.

TODO: compute observability matrix for speeds and generate plots


As the system is observable, we estimate the unmeasured states with a Kalman filter.
For this system, roll angle $ \phi $ and roll rate $ \dot{\phi} $ do not exist in the physical setup and can only be
obtained through estimation.
While it is possible to obtain the steer rate $ \dot{\delta} $ through a direct measurement with an angular rate
gyroscope, the bandwidth trade-off from estimation is minimal (TODO: compute and show this).
Given the system equations in \autoref{eq:ss}, we describe the Kalman filter model as the state space equations with
process and measurement noise:
\begin{equation}
\begin{aligned}
    \dstateEst &= \stateMat \stateEst + \inputMat \sysInput + \processNoise \\
    \meas &= \outputMat \stateEst + \measNoise \label{eq:kalman_ss}
\end{aligned}
\end{equation}
where $ \stateEst $ and $ \dstateEst $ denote the state estimate and time derivative of the state estimate,
$ \processNoise $ and $ \measNoise $ are realizations of the process and measurement noise respectively,
and $ \meas $ is used to explicitly distinguish the system measurement from the system output $ \sysOutput $.

Process noise $ \processNoise $ and measurement noise $ \measNoise $ are drawn from random processes with
Gaussian distributions and described by the
process noise covariance matrix $ \processCov $ and measurement noise covariance matrix $ \measCov $.
% TODO: mention that Q > 0 and R >= 0?
The noise covariance matrices are assumed constant when the system has reached steady-state
(i.e. $ v $ is not changing with time). % what about steady turning?

As a result, we obtain the optimal Kalman gain matrix $ \kalmanGain $
%and estimate covariance matrix $ \estimateCov $
parameterized by forward speed $ v $:
\begin{equation}
    \stateEst = \kalmanGain \meas
\end{equation}


\subsection{Kinematics}

In modeling, a number of coordinates used to describe the kinematics of the bicycle are independent of the stability, or
steer dynamics as described in \autoref{sec:steer_dynamics}.
The kinematics of the bicycle can computed with the following equations:


\subsection{Implementation}
key points to address:
\begin{itemize}
    \item v parameterizes the system/equations of motion
    \item we compute the discrete time state space with:
        \begin{equation}
        \exp(\begin{bmatrix} \stateMat & \inputMat \\ \bm{0} & \bm{0} \end{bmatrix}^T)
            = \begin{bmatrix} \stateMat_d & \inputMat_d \\ \bm{0} & \bm{I} \end{bmatrix}
        \end{equation}
    \item Need to discretize Kalman noise covariance matrices!
        https://en.wikipedia.org/wiki/Discretization
    \item matrix exponential computed using scaling and squaring with
        Pad{\'e} approximants\cite{higham2005} as implemented in Eigen\cite{eigenweb}
    \item kinematic coordinates computed with ode solvers as they are nonlinear
    \item states are discretized instead of integrated for efficiency\\
        computation of matrix exponential only once for each foward speed v\\
        ode integration is slower\cite{moler2003} (for computing the matrix exponential)
        note that ode integration for a \textit{single} time step may be slower, \\
        but there is no need to perform all the repeated integration steps as
        we obtain a difference equation. (plus measurement shows it to be faster)
\end{itemize}

\section{Hardware design?}

\section{Comparison of data to naturalistic studies}

\section{Conclusions}

\section{Acknowledgments}
We gratefully acknowledge the European Commission for their support of the Marie Curie Initial Training Network (ITN)
project Nr. 608092 MOTORIST (Motorcycle Rider Integrated Safety), www.motorist-ptw.eu.

\bibliography{references.bib}
\bibliographystyle{plain}
\end{document}

Note from Arend:
In this work we address the need for haptic feedback in the balancing task in bicycling. For that we use an experimental
bicycle simulator setup which has been described in Recuero and Schwab 2013. The system, which is shown in
Fig~\ref{fig1}, consists of a stationary bicycle on which one can pedal and steer. The visual feedback is by means of a
real time animation on a flat computer screen, with either a first or third person view. The steering assembly has
torque feedback from the computer model. In this bicycle simulator the rider interacts with a virtual environment,
receiving realistic handlebar torque from a bicycle model, which runs in the computer simulation model. The lateral
dynamics of the bicycle are governed by the linearized equations of motion for lean and
steer~\cite{meijaard2007linearized} as implemented in the computer model. The state of the handlebar, lean angle and
lean angle rate, are measured and together with the computer generated lean angle and its time derivatives used to
estimate handlebar haptic feedback torque $T_m$. The steer dynamics is governed by the real handlebar system. Stability
of the haptic system for any kind of human contribution, e.g., tight grasp or sudden release, must be guaranteed. In so
doing, one can resort to place a ``virtual coupling'' between the haptic device and the virtual environment that acts as
a mechanical filter~\cite{adams1999stable}. The bicycle haptic interface, as developed here, shows a impedance
casuality, i.e., forces are transmitted to the rider, whereas the input of the virtual environment is the handlebar
state. Uncertainties in the measurement of steering angle and its rate may lead, depending on the physical model
utilized, to unrealistic fed back torque or excessive phase lag~\cite{katzourakis2009design}. The set of sample rates of
sensors and the model is another key factor in the haptic system. It must be high enough to keep a pleasant refreshing
rate of the simulator visualization and results in natural and smooth feeling in rider's limbs.
