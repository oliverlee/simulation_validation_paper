\documentclass[11pt,a4paper]{amsart}
\usepackage{geometry}               % See geometry.pdf to learn the layout options. There are lots.
\geometry{left = 3cm, right = 3cm, top = 3cm, bottom = 3cm}
\usepackage[parfill]{parskip}       % Begin paragraphs with an empty line rather than an indent
\usepackage[colorlinks=false]{hyperref}

\frenchspacing                      % single space between sentences

\title{Design and validation of a fixed-base bicycle simulator}
%\author{Oliver Lee}

\begin{document}
\maketitle

\section{Introduction}
%% overview on cycling
%% address safety in cycling, understanding rider behavior
Cycling is a common form of transportation in a number of countries.
While cycling has long been prevalent as a cheap form of transportation in many developing countries,
there has been a resurgence in ridership for developed countries as a cheap form of travel,
not subject to the congestion of car traffic in large cities.
These have a shown a large number of single vehicle accident, thus a better understanding of the bicycle rider system is
beneficial in reducing these accidents and improving the safety of cyclists.

%% history on bicycle and rider research
While the bicycle has been used for transport since the 19th century\cite{wilson2004}, the body of work studying the
vehicle is smaller relative to motorized vehicles such as motorcycles, cars, and planes.
Whipple developed a model for the bicycle in 1899\cite{whipple1899} but it was not until 2007 that the model was
experimentally validated\cite{kooijman2008}.
Models, as well as real world experience, show the bicycle rider system to be unstable for a range of speeds;
active control from the rider is required rider to stabilize the bicycle.
Recent work in the field of bicycle dynamics has focused on identification of rider control
models by means of experimental data collection\cite{delange2011,hess2012,hladun2015}.
Data collected for lateral line tracking\cite{delange2011,hess2012}, lane change\cite{hladun2015}, and stabilization
with roll torque disturbances\cite{delange2011,hess2012} were used for system identification and parameter estimation of
the gains of the human controller.

%% why is a simulator important (motivation for building a simulator)?
% goal of system identification
There are always limitations with simulators;
certainly with a fixed-base, one is incapable of simulating the vestibular inputs.
However, a simulator with high enough fidelity provides a number of benefits compared to using instrumented bicycles.
Namely, reproducibility of experiments, ease of data collection, and improved safety in test environments.
As the rider bicycle system is unstable, simulators allow for tests of scenarios that have high risk of injury if
instrumented bicycles were to be used.

Studies on system identification of drivers using (car) driving simulators?

%% other simulators?
Talk to Natalia about other bicycle and motorcycle simulators.

There is little work focusing on human rider control using a bicycle simulator.
De Lange created a simulator in Matlab with the rider controlling the virtual bicycle by means of a
gamepad\cite{delange2011}.
His simulator only provides visual and not vestibular or proprioceptive feedback and finds its use limited for
analyzing rider control due to difficulty in controlling the bicycle.
As a precursor to this work, Schwab and Recuero developed a simulator with actuated handlebars\cite{schwab2013},
presenting the rider with visual and proprioceptive feedback.
They find haptic handlebar feedback torque necessary for the rider to stabilize the bicycle.

We begin with a description of the simulator.
Then a test procedure to evaluate simulator fidelity is presented.
Simulator data and data from naturalistic studies are compared and discussed.
Finally, we end with some conclusions and suggestions for future work.

\section{Simulator design}

\section{Comparison of data to naturalistic studies}

\section{Conclusions}

\section{Acknowledgments}
We gratefully acknowledge the European Commission for their support of the Marie Curie Initial Training Network (ITN)
project Nr. 608092 “MOTORIST” (Motorcycle Rider Integrated Safety), www.motorist-ptw.eu.

% include references that aren't explicitly cited
\nocite{meijaard2007}
\bibliography{references.bib}
\bibliographystyle{plain}
\end{document}
